
% ====
\newpage
\begin{rSection}{Research Experience - In University}

    {\bf \textcolor{CUHKblue}{PhD candidate, Carnegie Mellon University, United States}}               \hfill { Aug.~2021 -- Now} \\
    {\bf {Multi-modal agentic LLMs for EDA }}
    \begin{itemize}[noitemsep,topsep=-5pt]
        \item Introduces a new paradigm where \underline{EDA agents move beyond rigid, predefined APIs} and interact with a unified, \underline{multi-modal} representation of the design environment.
        \item Outlines two challenging case studies—automated \underline{RTL debugging} and \underline{logic diagnosis}—to demonstrate the framework's effectiveness.
        \item Investigates the \underline{new problem-solving strategies} that emerge from this increased agent autonomy. 
        % \item Study how this more complex interaction model impacts \underline{inference-time scaling laws}, particularly the efficiency of repeated sampling.
    \end{itemize} 
    % \list{1}{2}
    {\bf {DEFT: Differentiable Automatic Test Pattern Generation}}
    \begin{itemize}[noitemsep,topsep=-5pt]
        \item Reformulated ATPG as a \underline{differentiable optimization} with a new reparameterization
        \item Custom CUDA kernel achieves \underline{4$\times$–26$\times$} speedup on industrial circuits
        \item Improved HTD detection by \underline{21\%–49\%} under the same pattern budget
        \item Support practical ATPG features such as partial assignment, generating patterns with \underline{19.3\% fewer 0/1 bits} while detecting 35\% more faults
    \end{itemize}
    {\bf {Graph modality in LLMs \textbf{[{{ICLAD'25, Best Paper Honorable Mention Award}}]}}}
    \begin{itemize}[noitemsep,topsep=-5pt]
        \item Explore the \underline{graph modality integration into LLMs} for VLSI
        \item A fully automated data collection pipeline
        \item Collect more than \underline{10 billion} training tokens  
    \end{itemize} 
    {\bf {Global Floorplanning using Semidefinite Programming \textbf{[{{DAC'23}}]}}} 
    \begin{itemize}[noitemsep,topsep=-5pt]
        \item A SDP-based method for finding the \underline{best locations of modules} in a chip
        \item Reduced wirelength by 3.02\%--20.01\%
        \item The industrial case study shows \underline{\textbf{500\%}} quality improvement compared to the industrial tool.
    \end{itemize} 
    {\bf {Pseudo-Exhaustive Physically-Aware Region Testing \textbf{[{ITC'22}]}}}
    \begin{itemize}[noitemsep,topsep=-5pt]
        \item Comprehensively analyze both the physical layout and the logic netlist to identify single- or multi-output sub-circuits.
        \item Implemented a novel tensor-based representation of layout polygon coordinates that enables a neighborhood
        search strategy that reduces computational complexity
        from $O(n^2)$ to $O(dn)$.
        \item Implemented a \underline{GPU-based} algorithm the physical sub-circuit extraction containing billions of sub-circuits.
    \end{itemize}
    {\bf {GNN study in logic locking \textbf{[{MLCAD'23}]}}} 
    \begin{itemize}[noitemsep,topsep=-5pt]
        \item Modeled their ability to identify circuit
        changes that stem from a logic lock as the ability to {decide
        the isomorphism between logic netlists.}
        \item Showed that GNNs
        are always \underline{upper bounded by heterogeneous Weisfeiler Lehman
        test} in deciding the netlist isomorphism, and gave the conditions
        when GNNs reach the bound. \\
    \end{itemize}


{\bf \textcolor{CUHKblue}{MPhil Student, The Chinese University of Hong Kong, Hong Kong}}               \hfill { Aug.~2019 -- May.~2021} \\
% \list{1}{2}
{\bf {Routing Tree Construction \textbf{[{ASP-DAC'21, Best Paper Award}]}}}
\begin{itemize}[noitemsep,topsep=-5pt]
    \item \underline{Formalized special properties} of the point cloud for the routing tree construction with theoretical proof.
    \item Proposed an adaptive flow, which used the cloud embedding obtained by a {specifically-designed model based} \underline{on special properties}, to select the best approach and predict the best parameter.
    \item Outperformed previous methods by a large margin yet being extensible and flexible.
\end{itemize}
{\bf {Adaptive Layout Decomposition \textbf{[{DAC'20, TCAD'21}]}}}
\begin{itemize}[noitemsep,topsep=-5pt]
    \item Proposed an adaptive workflow for efficient \underline{decomposer selection} and \underline{graph matching} using \underline{graph embeddings}.
    \item Designed a \underline{graph library construction algorithm} based on graph embeddings for small graphs excluding isomorphic ones.
    \item Reduced the runtime by 87.7\% while still preserving the optimality compared with the ILP-based decomposer.\\
\end{itemize}

% {\bf {Reviewed paper for AAAI'22, about GNNs for graph coloring}\\ \\
{\bf \textcolor{CUHKblue}{Research Assistant, The Chinese University of Hong Kong, Hong Kong}}               \hfill { Feb.~2019 -- July.~2019} \\
{\bf {Open-source Layout Decomposition Framework \textbf{[{TCAD'21}]}}}
\begin{itemize}[noitemsep,topsep=-5pt]
    \item Presented an \underline{open-source layout decomposition framework}, with efficient implementations of various state-of-the-art simplification and decomposition algorithms.
    \item Discovered a set of issues of previous algorithms and proposed corresponding solutions.
\end{itemize}
{\bf {Acceleration and Compression of DNNs \textbf{[{ICTAI'19, Best Student Paper Award}]}}}
\begin{itemize}[noitemsep,topsep=-5pt]
    \item Proposed a unified framework to \underline{compress CNNs} by combining both lowrankness and sparsity.
    \item Compressed the model with up to 4.9$\times$ reduction of parameters at a cost of little loss.\\
\end{itemize}

{\bf \textcolor{CUHKblue}{Research Assistant, Southern University of Science and Technology, China}}  \hfill { June.~2018 -- Jan.~2019} \\
{\bf {Testing of Auto-driving Systems \textbf{[{ICSE'20}]}}}
\begin{itemize}[noitemsep,topsep=-5pt]
    \item Introduced a joint optimization method to systematically \underline{generate adversarial perturbations} to mislead steering of an autonomous driving system physically.
    \item Demonstrated the possibility of \underline{continuous physical-world tests} for auto-driving scenarios as the first study.
\end{itemize}
{\bf {Fault Localization \textbf{[{ISSTA'19, Distinguished Paper Award}]}}}
\begin{itemize}[noitemsep,topsep=-5pt]
    \item Proposed a \underline{hierarchical DL approach} to learn the most effective features for precise fault localization. 
    \item Significantly outperformed state-of-the-art with over 20\% improvement. 
\end{itemize}
\end{rSection}



% ====
\begin{rSection}{Research Experience - In Industry}

    {\bf \textcolor{CUHKblue}{Intern, SoC Physical Design Group, Apple, United States}}               \hfill { May.~2024 -- Aug.~2024} \\
    {\bf {Floorplan Encoder}}
    \begin{itemize}[noitemsep,topsep=-5pt]
        \item Propose a novel floorplan encoder for the floorplanning task, the encoder is capable of encoding the \underline{multi-modal} and \underline{multi-objects} floorplan state.
        \item Achieves 95\% accuracy and shows 3X speedup to achieve the same quality compared to the industrial tool.  \\
    \end{itemize}
    {\bf \textcolor{CUHKblue}{Research Intern, Nvidia, United States}}               \hfill { May.~2023 -- Aug.~2023} \\
    % \list{1}{2}
    {\bf {Differentiable Global Routing \textbf{[{{DAC'24}}]}}}
    \begin{itemize}[noitemsep,topsep=-5pt]
        \item A differentiable global router capable of concurrent \underline{optimization for millions of nets}
        \item Reduced nets with overflow by more than \underline{80\%}\\
    \end{itemize} 

{\bf \textcolor{CUHKblue}{Intern, SoC Physical Design Group, Apple, United States}}               \hfill { June.~2022 -- Sep.~2022} \\
% \list{1}{2}
{\bf {Exploration of GNNs for Physical Design }}
\begin{itemize}[noitemsep,topsep=-5pt]
    \item Implemented a basic GNN model for predicting holder buffer before routing.
    \item Evaluated multiple architectures, including path-based, sub-circuit based, and sub-graph based models.
\end{itemize}
{\bf {Perfect Rectilinear Floorplanning}}
\begin{itemize}[noitemsep,topsep=-5pt]
    \item A Simulated Annealing based algorithm for perfect rectilinear floorplanning.
    \item Reinforcement learning, and supervised-learning that guides SA are also explored.
    \item \underline{\textbf{{Integrated into physical design flows for large-scale SoC development.}}}\\
\end{itemize}
\end{rSection}
